\documentclass{article}
\usepackage[top=3cm, margin=1cm]{geometry}
\usepackage[german]{babel}

\title{myFind Protokoll}
\author{Paul Passauer, Samuel Hammerschmidt}
\date{November 2024}

\begin{document}

\maketitle

\section*{Konzept der Parallelisierung}
Das Programm soll nach Einlesen der Suchoptionen pro angegebenen Dateinamen einen Kindprozess erstellen. Diese suchen alle am selben Suchpfad, der durch den User angegeben wird, parallel nach den unterschiedlichen Dateinnamen.
Dies reduziert die Bearbeitungszeit des Programms. Durch die Verwendung von Message Queues senden die Kindprozesse die Ergebnisse an den Elternprozess, der diese dann ausgibt.

\section*{Ablauf der Parallelisierung}
\begin{enumerate}
    \item Der Elternprozess erstellt für jeden Dateinamen einen Kindprozess. Dies passiert in einer \texttt{foreach Schleife}, die über die Dateinamen iteriert und mit \texttt{fork()} einen Kindprozess erstellt.
    \item Über einen \texttt{Switch-Case} wird geprüft ob es sich um den Eltern- oder Kindprozess handelt.
    \item Sollte es beim Abspalten der Kindprozesse zu Problemen kommen kommt es zu \texttt{case: -1} (Error-Case). In diesem Fall wird die Message Queue gelöscht und eine Fehlermeldung wird ausgegeben.
    \item Die Kindprozesse arbeiten in \texttt{case: 0} und rufen darin die Funktion \texttt{search\_file()} auf, die im angegebenen Verzeichnis nach dem Dateinamen sucht.
    \item Der Elternprozess arbeitet in \texttt{case: default} und wartet durch \texttt{wait()} auf den Abschluss der Kindprozesse. Die Ergebnisse der Suche werden über eine Message Queue empfangen.
    \item Der Elternprozess gibt die Ergebnisse auf \texttt{stdout} aus.
\end{enumerate}

\section*{Verwendung von Message Queues}
Bei der Message Queue handelt es sich um eine Sammlung von Messages. Eine Message wird druch ein \texttt{Struct} repräsentiert, welches aus einem \texttt{long msg\_type} (verwendet für die Zuordnung von Kindprozess zu Nachricht) und einem \texttt{char msg\_text[100]} (der eigentliche Inhalt der Nachricht) besteht.
\begin{enumerate}
    \item Die Message Queue wird noch vor dem \texttt{fork()} für die Kindprozesse durch den zuvor erwähnten \texttt{foreach Loop} erstellt. Die Funktion \texttt{search\_file()} erhält die ID der Message Queue \texttt{msg\_id} als Parameter.
    \item Nach der Suche wird die Nachricht \texttt{msg} über die Message Queue mit \texttt{msgsnd()} an den Elternprozess gesendet. Dieser empfängt die Message mit \texttt{msgrcv()}.
    \item Um eine eindeutige Identifikation der Nachrichten und eine korrekte Reihenfolge zu garantieren, wird der \texttt{msg\_type} auf die PID des Kindprozesses gesetzt.
    \item Nachdem alle Kindprozesse ihre Ergebnisse an den Elternprozess gesendet haben (garantiert durch \texttt{wait()} im Elternprozess), gibt der Elternprozess die Ergebnisse auf \texttt{stdout} aus.
    Dies passiert durch eine \texttt{while-Schleife}, die über die Einträge der Message Queue iteriert und die Nachrichten ausgibt. Da die Kindprozesse nicht direkt mit \texttt{stdout} arbeiten, sondern über die Message Queue, hat letztendlich nur der Elternprozess die Möglichkeit, die Nachrichten über \texttt{stdout} auszugeben.
    \item Schließlich wird die Message Queue mit \texttt{msgctl()} gelöscht.
\end{enumerate}

\section*{Zusammenfassung}
Die Verwendung von Kindprozessen im Zusammenspiel mit dem \texttt{wait()} des Elternprozesses garantiert eine parallele Suche.
Die Message Queue sorgt für eine geordnete und mit Parallelisierung kompatible Möglichkeit für die Ausgabe der Ergebnisse auf \texttt{stdout}.

\end{document}